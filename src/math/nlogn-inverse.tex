\documentclass[a4paper, 12pt, fleqn]{article}

\usepackage{amsmath}
\usepackage{amsthm}
\usepackage{amssymb}
\usepackage{xcolor}
\usepackage{float}
\usepackage{hyperref}
\usepackage{url}
\usepackage[margin=1in]{geometry}

\hypersetup{
    colorlinks,
    linkcolor={red!50!black},
    citecolor={red!50!black},
    urlcolor={blue!50!black}
}

\newtheorem{theorem}{Theorem}
\newtheorem{definition}{Definition}
\newtheorem{example}{Example}
\newtheorem{corollary}{Corollary}[theorem]
\newtheorem{lemma}[theorem]{Lemma}

\newcommand*{\floor}[1]{\left\lfloor #1 \right\rfloor}
\newcommand*{\ceil}[1]{\left\lceil #1 \right\rceil}
\newcommand*{\abs}[1]{\left\lvert #1 \right\rvert}
\newcommand*{\norm}[1]{\left\lVert #1 \right\rVert}
\DeclareMathOperator*{\E}{E}
\DeclareMathOperator*{\Var}{Var}
\DeclareMathOperator*{\argmin}{argmin}
\DeclareMathOperator*{\argmax}{argmax}

\newcommand{\initMinimal}{
\setlength{\parindent}{0pt}
\setlength{\parskip}{0.5em}
}

\newcommand{\initFromContents}{
\tableofcontents
\newpage
\initMinimal{}
}

\newcommand{\initAfterBeginDocument}{
\maketitle
\initFromContents{}
}

\newcommand{\addMyBib}{
\bibliographystyle{plainurl}
\bibliography{bibdb}
}

\author{Eklavya Sharma}

\date{}


\title{Inverse of \texorpdfstring{$x \mapsto x\log_b x$}{x -> x*log_b(x)}}

\begin{document}

\maketitle
\initMinimal{}

Let $b > 1$. Let $f: \mathbb{R}_{\ge 0} \mapsto \mathbb{R}_{\ge 1}$ be a function such that
$f(x)\log_b(f(x)) = x$; i.e. it is the inverse of $x \mapsto x\log_b(x)$.
Our objective is to determine the asymptotic behavior of $f(x)$.

\begin{theorem}
$f$ is a bijective function.
\end{theorem}
\begin{proof}
Let $h: \mathbb{R}_{\ge 1} \mapsto \mathbb{R}_{\ge 0}$ where $h(x) = x\log_b x$.

$x \ge 1 \implies x \log_b x \ge 0$. So $h$ is a function.
$h'(x) = \log_b x + \frac{1}{\ln b} > 0$. So $h(x)$ is strictly increasing.

Since $h$ is a strictly increasing function from $[1, \infty)$ to $[0, \infty)$
and $h(1) = 0$, $h$ is a bijective function (proof beyond the scope of this document).

Since $f$ is the inverse of $h$, $f$ is also bijective
(proof beyond the scope of this document).
\end{proof}

Define functions $\ell$ and $u$ as
\begin{align*}
\ell(x) &= \frac{x}{\log_b x}
& u(x) &= \left(1 + \frac{1}{e\ln b}\right)\frac{x}{\log_b x}
\end{align*}

\begin{theorem}
$y \ge b \implies \ell(y) \le f(y)$
\end{theorem}
\begin{proof}
Let $y \ge b$ and $x = f(y)$. Therefore, $y = x\log_b x$ and $x \ge b$.
\begin{align*}
\ell(y) &= \ell(x\log_b x)
\\ &= \frac{x\log_b x}{\log_b x + \log_b\log_b x}
\\ &\le \frac{x\log_b x}{\log_b x}  \tag{$x\log_b x \ge 0$ and $\log_b \log_b x \ge 0$}
\\ &= x = f(y)
\end{align*}
\end{proof}


\begin{lemma}
\label{thm:lem1}
Let $g(x) = \frac{\log_b x}{x}$. Then $g(x) \le \frac{1}{e\ln b}$.
\end{lemma}
\begin{proof}
\begin{align*}
g(x) &= \frac{1}{\ln b} \frac{\ln x}{x}
& g'(x) &= \frac{1}{\ln b} \frac{1-\ln x}{x^2}
\end{align*}
\[ x \ge e \implies 1-\ln x \le 0 \implies g'(x) \le 0 \]
\[ 0 < x \le e \implies 1-\ln x \ge 0 \implies g'(x) \ge 0 \]
Therefore, $g(x)$ attains its maximum value at $x = e$.
\[ g(x) \le g(e) = \frac{1}{e\ln b} \]
\end{proof}

\begin{theorem}
$y \ge 1 \implies f(y) \le u(y)$
\end{theorem}
\begin{proof}
Let $y \ge 1$ and $x = f(y)$. Therefore, $y = x\log_b x$ and $x \ge b$.
Let $\frac{1}{e\ln b} = c$.
\begin{align*}
u(y) &= u(x\log_b x)
\\ &= (1+c)\frac{x\log_b x}{\log_b x + \log_b \log_b x}
\\ &= (1+c)\frac{x}{1 + g(\log_b x)}
\\ &\ge x \tag{$x \ge 0$ and $g(\log_b x) \le c$}
\\ &= f(y)
\end{align*}
\end{proof}

Therefore,
\[ f(x) \in \Theta\left(\frac{x}{\log_b x}\right) \]

%\addMyBib{}

\end{document}
